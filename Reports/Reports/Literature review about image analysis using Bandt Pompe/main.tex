\documentclass[12pt]{article}
\usepackage{sbc-template}
\usepackage[utf8]{inputenc}
\usepackage[portuguese]{babel}
\usepackage{lipsum} 
\usepackage{graphicx,url}
\usepackage{amsmath,bm,bbm} 
\usepackage[]{subfigure}
\usepackage{natbib}
\sloppy

\title{Revisão de literatura sobre análise de imagens utilizando teoria da informação}

\author{Roger Matos\inst{1}}

\date{Abril, 2019}

\address{
  Laboratório de Computação Científica e Análise Numérica (LaCCAN)\\
  Universidade Federal de Alagoas (UFAL) -- Maceio, AL -- Brazil
}

\begin{document}

\maketitle

\section{Introdução}

Este relatório descreve o processo de revisão sistemática de literatura acerca da análise de imagens utilizando a técnica de Bandt \& Pompe. Será abordada a metodologia utilizada e logo depois, os resultados obtidos, finalizando com uma conclusão explicitando a importância dos artigos obtidos durante esse processo.

\section{Metodologia}

A portal de períodicos CAPES foi utilizado na busca dos artigos, diversas combinações de palavras chaves foram utilizadas, entre elas: "IMAGE + BANDT", "IMAGE ANALYSIS + INFORMATION THEORY", "TWO-DIMENSIONAL + BANDT + POMPE", "BRODATZ + BANDT + POMPE" e "BRODATZ + ENTROPY + COMPLEXITY", uma vez que foi solicitado um foco em encontrar análises de texturas Brodatz que utilizassem também a metodologia de Bandt \& Pompe. Ademais, o artigo \textit{History of art paintings through the lens of entropy and complexity} que vinha sendo utilizado no projeto serviu como uma base de comparação para o que vinha sendo buscado.

\section{Resultados obtidos}

Foi produzido um arquivo BibTex com os artigos selecionados, todos atendendo ao critério de utilizar a metodologia de Bandt \& Pompe em análise de imagens tal qual o artigo citado acima. Contudo, não foram encontrados artigos que fizessem uma extensiva análise de texturas Brodatz utilizando esse método, apenas um, \textit{Discriminating image textures with the multiscale two-dimensional complexity-entropy causality plane}, lida com isso, mas somente em um tópico do mesmo, não sendo o foco principal do artigo. 

\section{Conclusão}

Esses artigos poderão ser importantes referências para o progresso do projeto em andamento, uma vez que a análise de texturas em imagens SAR com teoria da informação está sendo trabalhada a fim de produzir novo conhecimento científico acerca desse assunto. Além disso, essa revisão de literatura mostra também que a análise de texturas Brodatz utilizando Bandt \& Pompe ainda não foi trabalhada a fundo e demonstra uma possível área a ser trabalhada futuramente.

\end{document}

