\documentclass[10pt]{article}

\usepackage{../../Common/sbc-template}
\usepackage[utf8]{inputenc}
\usepackage[portuguese]{babel}
\usepackage{lipsum} 
\usepackage{graphicx,url}
\usepackage{amsfonts,amsmath,bm,bbm}
\usepackage{natbib}
\usepackage{esvect}
\usepackage[colorlinks=true, allcolors=blue]{hyperref} 

\sloppy

\title{Estudo de padrões ordinais com grafos de transição e Cadeias de Markov}


\author{
	Eduarda T.\ C.\ Chagas\inst{1},
	Heitor S.\ Ramos\inst{1},
	Alejandro C.\ Frery,
	Osvaldo A.\ Rosso
}

\address{Departamento de Ciência da Computação\\
  Universidade Federal de Minas Gerais (UFMG) -- Belo Horizonte, MG -- Brazil
  \email{eduarda.chagas@dcc.ufmg.br}}


\begin{document}

\maketitle

\section*{\centering Definição do problema} \label{abstract}

Uma das limitações das técnicas de análise de séries temporais baseadas em padrões ordinais é não permitirem fazer previsões (\textit{forecast}) nem simulações.
Este trabalho explora essa possibilidade.

Sejam $\bm Z=(Z_1,\dots,Z_n)$ uma série temporal de valores reais,
$\bm \pi^{(D,\tau)} = (\pi^{(D,\tau)}_1,\dots,\pi^{(D,\tau)}_{N-(D-1)\tau})$ a série de padrões a ela associados (calculados com palavras de dimensão $D$ e atraso $\tau$),
e $G=(V,E)$ o grafo de transições obtido a partir de $\bm \pi^{(D,\tau)}$.

A proposta consiste em fazer a junção das evidências de $\bm Z$, $\bm \pi^{(D,\tau)}$, e $G$ para fazer simulações e previsões a respeito da série.

O primeiro passo consiste em analisar $\bm Z$ e $\bm \pi^{(D,\tau)}$.
Para cada padrão observado em $\bm \pi^{(D,\tau)}$, serão coletados os dados que o originaram em $\bm Z$.
Consideremos, por exemplo, o caso do padrão $\pi^{(3,1)}_1=b_{j}b_{j+1}b_{j+2}$, e suponhamos que ele corresponde a todas as palavras que satisfazem $z_{j}<z_{j+1}<z_{j+2}$.
Todas essas palavras serão coletadas e analisadas para obter:
\begin{itemize}
	\item uma estimativa da distribuição três-variada, ou
	\item estimativas do valor central e estimativas de uma medida de dispersão de cada um dos três valores, por exemplo a média e o desvio padrão.
\end{itemize} 
Teremos, assim, associados ao padrão $\pi^{(3,1)}_1$, 
\begin{itemize}
	\item um modelo $\widehat{\mathcal{D}}(\pi^{(3,1)}_1)$, ou
	\item três médias $\widehat\mu_{b_j}, \widehat\mu_{b_{j+1}}, \widehat\mu_{b_{j+2}}$ e três desvios padrão $s_{b_j}, s_{b_{j+1}}, s_{b_{j+2}}$.
\end{itemize} 

O segundo passo consiste em formar a matriz de transições do grafo $G$, digamos $M$.
Por construção, a cadeia é irredutível, e basta com que haja uma única transição entre estados iguais para que a cadeia seja aperiódica.
Com estas propriedades, há uma única distribuição de equilíbrio $\Pi$, que é a solução de $\Pi M=\Pi$.

Dada a série temporal $\bm Z=(Z_1,\dots,Z_n)$, associada à sequência $\bm \pi^{(D,\tau)} = (\pi^{(D,\tau)}_1,\dots,\pi^{(D,\tau)}_{N-(D-1)\tau})$ de padrões, simularemos o evento $Z_{n+1}$ com dois elementos:
\begin{itemize}
	\item o padrão $\pi^{(D,\tau)}_1,\dots,\pi^{(D,\tau)}_{N-(D-1)\tau+1}$ que possui probabilidade máxima de ocorrência em $\Pi$ após o último padrão, e
	\item uma observação do modelo $\mathcal D$ correspondente a esse padrão. Note-se que será necessário obter apenas uma amostra da distribuição marginal de $\mathcal D$ dadas as observações já presentes nos últimos estágios da série.
\end{itemize} 

A nossa previsão da observação que sucede $Z_n$ será o estado de equilíbrio mais plausível que segue ao último padrão que inclui $Z_n$, e em cada posição do padrão colocaremos a estimativa de centralidade, munida da sua estimativa de precisão.
De forma mais sofisticada, poderemos usar o algoritmo EM~\citep{dempster_em}).

\bibliographystyle{agsm}
\bibliography{../../Common/references.bib}

\end{document}