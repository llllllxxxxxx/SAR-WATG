\documentclass[12pt]{article}
\usepackage{sbc-template}
\usepackage[utf8]{inputenc}
\usepackage[portuguese]{babel}
\usepackage{lipsum} 
\usepackage{graphicx,url}
\usepackage{amsmath,bm,bbm} 
\usepackage[]{subfigure}
\usepackage{natbib}
\sloppy

\title{Revisão de literatura sobre análise de imagens SAR utilizando texturas}

\author{Roger Matos\inst{1}}

\date{Abril, 2019}

\address{
  Laboratório de Computação Científica e Análise Numérica (LaCCAN)\\
  Universidade Federal de Alagoas (UFAL) -- Maceio, AL -- Brazil
}

\begin{document}

\maketitle

\section{Introdução}

Este relatório descreve o processo de revisão sistemática de literatura acerca da análise de imagens SAR através de suas texturas. Será abordada a metodologia utilizada e logo depois, os resultados obtidos, finalizando com uma conclusão explicitando a importância dos artigos obtidos durante esse processo.

\section{Metodologia}

Através da base \textit{Journal Citation Reports}(JCR), foram obtidos os dez melhores períodicos da categoria \textit{Remote Sensing} do ano de 2017, o mais recente na base. Com isso, foi visto que os seguintes periódicos possuíam as dez primeiras colocações:

\begin{enumerate}
    \item Remote sensing of environment;
    \item ISPRS journal of photogrammetry and remote sensing;
    \item IEEE geoscience and remote sensing magazine;
    \item GPS solutions;
    \item IEEE transactions on geoscience and remote sensing;
    \item Journal of geodesy;
    \item International journal of applied earth observation and geoinformation;
    \item Remote sensing;
    \item Photogrammetric engineering and remote sensing;
    \item IEEE geoscience and remote sensing letters.
\end{enumerate}

Daí, o portal de períodicos CAPES foi utilizado para fazer uma profunda busca em cada um desses citados acima. As combinações de palavras-chave utilizadas na busca foram: "SAR + TEXTURE + ANALYSIS", "SAR + TEXTURE + CLASSIFICATION", "SAR + TEXTURE + CHARACTERIZATION" e "SAR + TEXTURE".

\section{Resultados obtidos}

Um arquivo BibTex com um bom número de artigos foi produzido, todos com diversas técnicas de análise de imagens SAR utilizando suas texturas, acima de cada artigo foram explicitadas técnicas utilizadas. Todos tratam análise dessas imagens de forma genérica, classificando diferentes tipos de solo através dela.

\section{Conclusão}

Esses artigos serão boas referências para o projeto em andamento, uma vez que este se trata da análise de imagens SAR através de suas texturas e utilizando Teoria da Informação, serão úteis para comparações de resultados e de técnicas, entre outras coisas, podendo assim render novo conhecimento científico.

\end{document}
